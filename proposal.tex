\documentclass[10pt]{article}

\usepackage[margin=1in]{geometry}

\begin{document}

\begin{center}    
    \textbf{\Large 
    Bio-Inspired Traffic Flow: Using Penguin Huddle Models to Optimize Traffic Jam Dynamics
    }\\
    
    \vspace{0.5cm}

    Ana Costa$^1$, Zachary Caterer$^1$\\
    
    \vspace{0.5cm}
    
    $^1$Department of Chemical and Biological Engineering, University of Colorado Boulder\\
\end{center}


\section*{Introduction}
    Traffic congestion at uncontrolled intersections presents a challenge in urban planning. Aside from leading to energy innefficiencies, traffic congestion
increases emissions and driver fatigue. In recent years, adaptive cruise control (ACC) has enhaced the driving experience by relying on vehicles' automated
systems to perform simpler tasks such as maintaining speed and automatically breaking when necessary, without the need of driver input. Automated traffic congestion decision-making could
potentially be implemented into cruise control by adapting the collective decision making model of penguin huddles. Emperor penguins collectively huddle to optimize warmth and minimize exposure, while
simulataneously avoiding collisions to maintain a net benefit. The number of penguins that participate in huddles can reach the hundreds, which highlights
the effectiveness of the huddle behavior in individual decision-making for collective movement (Zitterbart et al.). This project aims to levarage the dynamics of penguin huddles to optimize traffic flow at uncontrolled intersections,
where congestion is prominent, by treating cars as agents that self-organize dynamically. 

\section*{Background and Related Work}
    The movement of emperor penguins in a huddle follows rules that balance individual needs with group dynamics. Individuals make decisions based on signals from their local
environment in order to maintain the required proximity for collective thermal regulation. Similarly, vehicles at an unsignalized intersection make local decisions to avoid 
collisions and minimize wait times. Prior research in agent-based modeling (ABM) and cellular automata (CA) has shown promise in simulating traffic flow under decentralized decision-making (Schäfer, Rico, et al).

\section*{Project Plan and Milestones}
We propose that implementing penguin-huddle dynamics into an ACC system will optimize the efficiency of vehicle movement in congested traffic scenarios.
 To develop a robust model, we will

\begin{itemize}
    \item Week 1-2: Review literature on penguin huddle dynamics and agent-based traffic models.
    \item Week 2-3: Formulate a mathematical model using partial differential equations (PDEs) and cellular automata.
    \item Week 3-4: Implement the model in Python using Mesa (for ABM), NumPy/SciPy (for PDEs), and Matplotlib (for visualization).
    \item Week 4-5: Simulate and analyze traffic efficiency with and without the huddle-inspired ACC system.
    \item Week 5-6: Finalize report and visualization of findings.
\end{itemize}

\section*{Mathematical and Computational Approach}

\subsection*{Agent-Based Modeling}
    Each car is treated as an agent with local decision-making. Rules for yielding, advancing, and stopping will be defined based on distance to other vehicles, similar to penguin huddle movement.
Once an agent reaches a certain distance from another at slow enough speeds, the ACC mode will be activated, and the agent will follow the huddle-inspired rules.

\subsection*{Partial Differential Equations and Cellular Automata}
We will explore:
\begin{itemize}
    \item PDE models adapted from heat transfer equations used in penguin huddling.
    \item Discrete cellular automata models where each cell represents a section of the intersection.
\end{itemize}

\section*{Computational Implementation}
\textbf{Programming Language:} Python or Julia 

\vspace{0.25cm}
\textbf{Libraries:}
\begin{itemize}
    \item Mesa for agent-based modeling
    \item NumPy/SciPy for PDE-based simulations
    \item Matplotlib for visualization
\end{itemize}

\textbf{Outputs:}
\begin{itemize}
    \item Heatmaps of traffic density
    \item Time evolution of queue lengths
    \item Comparisons with traditional traffic control methods
\end{itemize}

\section*{Impact and Conclusion}
This project aims to demonstrate whether bio-inspired self-organization can improve uncontrolled intersection efficiency in congested traffic scenarios. 
Understanding these dynamics could provide insights for urban planning and autonomous vehicle navigation in decentralized traffic systems.

\section*{Bibliography}
\begin{itemize}
    \item Zitterbart, Daniel P., et al. "Coordinated Movements Prevent Jamming in an Emperor Penguin Huddle." PLOS ONE, vol. 6, no. 6, 2011, e20260. \newline Public Library of Science, https://doi.org/10.1371/journal.pone.0020260.
    \item Schäfer, Rico, et al. "An Overview of Agent-Based Traffic Simulators." arXiv, 15 Feb. 2021, \newline ar5iv.labs.arxiv.org/html/2102.07505.
    \item Schadschhneider, A., et al. ``Traffic and Granular Flow." Springer, 2013.
    \item Moussaid, M., et al. ``How simple rules determine pedestrian behavior and crowd disasters." Proceedings of the National Academy of Sciences, 2011.
    \item Helbing, D. ``Traffic and Related Self-Driven Many-Particle Systems." Reviews of Modern Physics, 2001.
    \item Gu, F., et al. ``Emergence of collective oscillations in massive human crowds." Nature, 2025. 
\end{itemize}


\end{document}